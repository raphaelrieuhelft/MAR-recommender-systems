\documentclass[11pt, openany, a4paper]{article}

\usepackage{etex}
\usepackage{fullpage}
\usepackage{pstricks,pstricks-add,pst-math,pst-xkey}
\usepackage[frenchb]{babel}
%\usepackage{slashbox}
\usepackage{graphicx}
\usepackage{amsmath,amssymb,amstext,amsthm}
%\usepackage{comment}
\usepackage[utf8]{inputenc}
\usepackage[OT1]{fontenc}
\usepackage{pgf,tikz}
\usepackage{pgfplots}
\usepackage{floatpag}
\usepgfmodule{shapes}
\usetikzlibrary{arrows,patterns}
\usepackage{floatflt}
\usepackage{import}
\usepackage{xcolor}
%\usepackage{fourier-orns}

\newcounter{moncompteur}
\newtheorem{q}[moncompteur]{ \textbf{Question}}{}
\newtheorem{prop}[moncompteur]{ \textbf{Proposition}}{}
\newtheorem{df}[moncompteur]{ \textbf{Définition}}{}
\newtheorem*{df*}{ \textbf{Définition}}{}
\newtheorem{rem}[moncompteur]{ \textbf{Remarque}}{}
\newtheorem{theo}[moncompteur]{ \textbf{Théorème}}{}
\newtheorem{conj}[moncompteur]{ \textbf{Conjecture}}{}
\newtheorem{cor}[moncompteur]{ \textbf{Corollaire}}{}
\newtheorem{lm}[moncompteur]{ \textbf{Lemme}}{}
%\newtheorem{nota}[moncompteur]{ \textbf{Notation}}{}
%\newtheorem{conv}[moncompteur]{ \textbf{Convention}}{}
\newtheorem{exa}[moncompteur]{ \textbf{Exemple}}{}
\newtheorem{ex}[moncompteur]{ \textbf{Exercice}}{}
%\newtheorem{app}[moncompteur]{ \textbf{Application}}{}
%\newtheorem{prog}[moncompteur]{ \textbf{Algorithme}}{}
%\newtheorem{hyp}[moncompteur]{ \textbf{Hypothèse}}{}
\newenvironment{dem}{\noindent\textbf{Preuve}\\}{\flushright$\blacksquare$\\}
\newcommand{\cg }{[\kern-0.15em [}
\newcommand{\cd}{]\kern-0.15em]}
\newcommand{\R}{\mathbb{R}}
\newcommand{\K}{\mathbb{K}}
\newcommand{\N}{\mathbb{N}}
\newcommand{\Z}{\mathbb{Z}}
\newcommand{\C}{\mathbb{C}}
\newcommand{\U}{\mathbb{U}}
\newcommand{\Q}{\mathbb{Q}}
\newcommand{\B}{\mathbb{B}}
\newcommand{\card}{\mathrm{card}}
\newcommand{\norm}[1]{\left\lVert#1\right\rVert}
\pgfplotsset{compat=newest}
\newcommand{\La}{\mathcal{L}}
\newcommand{\Ne}{\mathcal{N}}
\newcommand{\D}{\mathcal{D}}
\newcommand{\Ss}{\textsc{safestay}}
\newcommand{\Sg}{\textsc{safego}}
\newcommand{\M}{\textsc{move}}
\newcommand{\E}{\mathcal{E}}
\newcommand{\V}{\mathcal V}
\setlength{\parindent}{0pt}
\newcommand{\myrightleftarrows}[1]{\mathrel{\substack{\xrightarrow{#1} \\[-.6ex] \xleftarrow{#1}}}}
\newcommand{\longrightleftarrows}{\myrightleftarrows{\rule{1cm}{0cm}}}

\definecolor{bleuclair}{rgb}{0.75,0.75,1.0}
\newcommand{\ANNOT}[1]{
  ~\linebreak
  \centerline{
    %{\Huge{\danger}}
    \large\fcolorbox{black}{bleuclair}{
      \begin{minipage}[h]{.8\linewidth}
      #1
      \end{minipage}
    }
    %{\Huge{\danger}}
  }
}

\newcommand\tikzmark[1]{%
  \tikz[overlay,remember picture,baseline] 
  \node[anchor=base](#1){};}

\newcommand\MyLine[3][]{%
  \begin{tikzpicture}[overlay,remember picture]
    \draw[#1] (#2.north west) -- (#3.south east);
  \end{tikzpicture}}


\graphicspath{{.}}
\newcommand{\e}[1]{$\times 10^{#1}$}
\begin{document}

\section*{Introduction}

Définition CF

Attentes (précision/temps de réponse/ajout rapide d'un rating/cold start...)

Efficacité peut varier selon la forme du jeu de données (sparsity...)


\section{Les algorithmes utilisés}
	\subsection{Algorithmes témoins}
		Algorithmes naïfs servant de point de comparaison.
	\subsection{SVD}
		Variantes sur la question bonus du DM (comment traiter les trous dans la matrice, avec ou sans traitement des biais...).
	\subsection{Algorithmes Slope One}
		Idée : régression linéaire, en imposant que la pente vaut 1.
		
		Ajout d'un nouveau rating et traitement de requêtes très rapides, il faut voir à quel point c'est moins précis que d'autres algos plus sophistiqués.
	\subsection{Algorithmes par similarité cosinus}
		Implémentation de l'algorithme vu en cours
		
	\subsection{Analyse en composantes principales : algorithme Eigentaste}
		Idée : s'appuyer sur un petit sous-ensemble d'objets notés par tous les utilisateurs (\emph{gauge set}) pour projeter un utilisateur sur un espace de petite dimension puis estimer ses notes à partir de celles de ses voisins au sens d'un algorithme de clustering.
		
		Défauts : on demande à un nouvel utilisateur de noter l'intégralité du gauge set pour l'ajouter, et le cold start pose problème.
		
		La précision est-elle vraiment meilleure que celle d'autres algorithmes plus simples ?
		
		
		
\section{Observations expérimentales}
	\subsection{Jeux de données}
		\begin{itemize}
			\item{Matrice de la question bonus du DM (pleine, on observe seulement une certaine fraction des ratings)}
			\item{Jeu de données Jester (ratings d'une centaine de blagues, dix blagues sont notées par toutes les utilisateurs) utilisé pour Eigentaste.}
			\item{Jeu de données plus grand (MovieLens) pour mesurer les difficultés liées aux temps d'exécutions dans des conditions plus réalistes ?}
		\end{itemize}
	\subsection{Mesures d'erreur}
		
	\subsection{Temps d'exécution}
\section*{Conclusion}


\end{document}